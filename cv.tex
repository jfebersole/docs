\documentclass[11pt]{article}
\usepackage[margin=1in]{geometry}
\usepackage{enumitem}
% \usepackage{hyperref}
\usepackage{titlesec}
\usepackage{amsmath}
\usepackage{setspace}
% \usepackage[hidelinks]{hyperref}
\usepackage{xcolor}
\definecolor{myblue}{HTML}{3B628C} % define from hex
\usepackage[colorlinks=true, urlcolor=myblue]{hyperref}

\usepackage[normalem]{ulem} % gives underlining macros


\setlength{\parindent}{0pt}
\setlength{\parskip}{6pt}
\pagenumbering{gobble}

% Section formatting
\titleformat{\section}{\large\scshape\raggedright}{}{0em}{}[\titlerule]
\titleformat{\subsection}[runin]{\bfseries}{}{0em}{}[.]

\newcommand{\header}[2]{
    \begin{center}
        {\LARGE \textbf{#1}} \\
        \vspace{10pt}
        #2
    \end{center}
}

\begin{document}

\header{JACOB EBERSOLE}{
  \href{https://jfebersole.com}{jfebersole.com} \,|\, \href{mailto:je696@georgetown.edu}{je696@georgetown.edu}  \,|\, (802) 777-7716
}


\section*{Education}

\textbf{Georgetown University}, Washington, DC \\
Ph.D., Economics \hfill Expected 2026 \\
M.A., Economics \hfill 2022

\textbf{Dartmouth College}, Hanover, NH \\
B.A., Economics and Environmental Studies \hfill 2014

\section*{Fields}
Applied Microeconomics, Public Economics, Environmental Economics
\section*{Research Experience}

\textbf{Research Assistant}, Professor Laurent Bouton, Georgetown University \hfill 2023–

\section*{Teaching Experience}

\textbf{Graduate Teaching Assistant}, Georgetown University

% Public Sector Economics, Environmental Economics, Senior Thesis Seminar in Political Economy, Empirical Applications in Political Economy, Microeconometrics, Economic Statistics, International Economics.

Empirical Applications in Political Economy \hfill 2025 \\
Public Sector Economics \hfill 2024 \\
Microeconometrics (Master's) \hfill 2023 \\
Senior Thesis Seminar in Political Economy \hfill 2023 \\
International Economics \hfill 2022 \\
Environmental Economics \hfill 2022 \\
Economic Statistics \hfill 2021

%   \item ECON 4433: Public Sector Economics (Spring 2024, Fall 2024)
%   \item ECON 5861: Microeconometrics (Master’s Level) (Fall 2023)
%   \item PECO 401: Senior Thesis Seminar in Political Economy (Spring 2023)
%   \item ECON 242: International Economics (Fall 2022)
%   \item ECON 275: Environmental Economics (Spring 2022)
%   \item ECON 121: Economic Statistics (Fall 2021)


\section*{Work Experience}

\textbf{Senior Research Analyst}, Industrial Economics, Inc. (IEc) \hfill 2014–2020

% OPTIONAL SECTIONS (UNCOMMENT AS NEEDED)
% \section*{Job Market Paper}
% \textbf{Title of Job Market Paper} \\
% Abstract: Brief description or link to the paper.

\section*{Academic Service}

\textbf{Co-Chair}, Economics Graduate Student Organization, Georgetown University \hfill 2023–2024

\section*{Job Market Paper}

\textbf{WIMBY: Wind in My Back Yard?} \\
Abstract: This paper examines how local costs and benefits shape political support for wind energy development in the United States. While wind projects generate substantial public health and climate benefits, they also impose concentrated local costs that can lead to opposition and blocked projects. Using data on proposed wind projects in Illinois, I estimate that environmental benefits exceed local property value losses by more than a factor of thirty, highlighting the inefficiency of project rejections. To better understand the political dynamics, I link spatial variation in local costs and benefits to precinct-level election results for the county officials responsible for project approval. I find that incumbents lose vote share in precincts that incur property value losses, but gain support in precincts that receive property tax revenues. These findings underscore the political challenges of renewable energy deployment in a decentralized regulatory system and point to the potential for policies that better align local incentives with national climate goals.

\section*{Works in Progress}
\textbf{Negotiated Growth: Housing Development and Discretionary Permitting in Boston}

% \textbf{Does the National Environmental Policy Act (NEPA) Protect the National Environment?}

% \section*{Conferences and Presentations}
% Title of paper, Conference Name, Year

\section*{Technical Skills}
Python, Stata, R, GIS, Causal Inference

% \section*{References}
% Available upon request. Or list 3–4 academic references with full contact info.

\end{document}
