\documentclass[11pt]{article}
\usepackage[margin=1in]{geometry}
\usepackage{enumitem}
\usepackage{titlesec}
\usepackage{amsmath}
\usepackage{setspace}
\usepackage{xcolor}
\definecolor{myblue}{HTML}{3B628C} % define from hex
\usepackage[colorlinks=true, urlcolor=myblue]{hyperref}
\usepackage[normalem]{ulem} % gives underlining macros


\setlength{\parindent}{0pt}
\setlength{\parskip}{6pt}
\pagenumbering{gobble}

% Section formatting
\titleformat{\section}{\large\scshape\raggedright}{}{0em}{}[\titlerule]
\titleformat{\subsection}[runin]{\bfseries}{}{0em}{}[.]

\newcommand{\header}[2]{
    \begin{center}
        {\LARGE \textbf{#1}} \\
        \vspace{10pt}
        #2
    \end{center}
}

\begin{document}

\header{JACOB EBERSOLE}{
  Georgetown University $\cdot$ Department of Economics \\
  3700 O St. NW $\cdot$ Washington, DC $\cdot$ 20057 \\
  \vspace{6pt}
  \href{https://jfebersole.com}{jfebersole.com} \,|\, 
  \href{mailto:je696@georgetown.edu}{je696@georgetown.edu} \,|\, 
  \href{tel:+18027777716}{(802) 777-7716}
}

\section*{Education}

\textbf{Georgetown University}, Washington, DC \\
Ph.D., Economics \hfill Expected 2026 \\
M.A., Economics \hfill 2022

\textbf{Dartmouth College}, Hanover, NH \\
B.A., Economics and Environmental Studies \hfill 2014

\section*{Fields}
Applied Microeconomics, Public Economics, Environmental Economics
\section*{Research Experience}

\textbf{Research Assistant}, Professor Laurent Bouton, Georgetown University \hfill 2023–

\section*{Teaching Experience}

\textbf{Graduate Teaching Assistant}, Georgetown University

Environmental Economics \hfill Fall 2025, Spring 2022 \\
Empirical Applications in Political Economy \hfill Spring 2025 \\
Public Sector Economics \hfill Fall 2024, Spring 2024 \\
Microeconometrics (Master's) \hfill Fall 2023 \\
Senior Thesis Seminar in Political Economy \hfill Spring 2023 \\
International Economics \hfill Fall 2022 \\
Economic Statistics \hfill Fall 2021

\section*{Work Experience}

\textbf{Senior Research Analyst}, Industrial Economics, Inc. (IEc) \hfill 2014–2020

\section*{Academic Service}

\textbf{Co-Chair}, Economics Graduate Student Organization, Georgetown University \hfill 2023–2024

\section*{Job Market Paper}

\textbf{WIMBY: Wind in My Back Yard} \\
Wind energy projects in Illinois generate global environmental benefits that exceed local property value losses by more than a factor of thirty. Yet county governments often reject proposed projects. To assess the electoral incentives of permit-issuing county officials, I link spatial variation in local costs and benefits to precinct-level election results. Following approvals, incumbent county officials lose vote share in precincts that incur property value losses, but gain votes in precincts that benefit from higher school district property tax revenues.
\section*{Works in Progress}
\textbf{The Price of Approval: Discretionary Review and Housing Supply in Boston} \\
New housing developments of 50,000 square feet or more in Boston trigger Large Project Review, a discretionary approval process requiring impact studies, public hearings, and negotiated community benefits. Developers frequently size projects just below this threshold to avoid review. Based on the extent of project bunching, I estimate that review adds nearly \$3 million in costs per affected project and has reduced housing delivered by large projects by at least 4 percent over the past two decades. The community benefits negotiated through review are small relative to both the compliance costs borne by developers and the property-tax revenue the city forgoes when projects are scaled down or never built. 

% \textbf{Does the National Environmental Policy Act (NEPA) Protect the National Environment?}

% \section*{Conferences and Presentations}
% Title of paper, Conference Name, Year

\section*{Technical Skills}
Python, Stata, R, GIS, Causal Inference

% \section*{References}

\end{document}
